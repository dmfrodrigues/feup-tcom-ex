\setcounter{section}{2}
\section{Preparation Activity 3 - NFAs}
{
\renewcommand{\thesubsection}{\thesection.\alph{subsection}}
\subsection{Item a}
The presented FA is a NFA, because from state $1$ and having as input $a$ the automaton can transition to $2$ or stay in $1$.
\subsection{Item b}
\begin{center}
\begin{minipage}[c]{0.25\textwidth}
	\begin{alignat*}{2}
		NFA    &= (Q_N, \Sigma, \delta_N, 1, F_N)\\
		Q_N    &= \{1,2,3\}\\
		\Sigma &= \{a,b,c,d\}\\
		F_N    &= \{3\}\\
		\delta_N &\colon Q_N \times \Sigma \rightarrow \mathscr{P}(Q_N)
	\end{alignat*}
\end{minipage}%
\begin{minipage}[c]{0.25\textwidth}
	\begin{alignat*}{2}
		\delta_N(1,a) &= \{1,2\}\\
		\delta_N(1,b) &= \{1\}\\
		\delta_N(1,c) &= \{1\}\\
		\delta_N(1,d) &= \{1\}\\
		\delta_N(2,a) &= \{2\}\\
		\delta_N(2,b) &= \{2,3\}
	\end{alignat*}
\end{minipage}%
\begin{minipage}[c]{0.25\textwidth}
	\begin{alignat*}{2}
		\delta_N(2,c) &= \{2\}\\
		\delta_N(2,d) &= \{2\}\\
		\delta_N(3,a) &= \{3\}\\
		\delta_N(3,b) &= \{3\}\\
		\delta_N(3,c) &= \{3\}\\
		\delta_N(3,d) &= \{3\}
	\end{alignat*}
\end{minipage}
\end{center}
\subsection{Item c}
\begin{center}
\begin{tabular}{r | c c c c}
	$     \delta_N$ & $a      $ & $b      $ & $c    $ & $d    $ \\ \hline
	$\rightarrow 1$ & $\{1,2\}$ & $\{1\}  $ & $\{1\}$ & $\{1\}$ \\
	$            2$ & $\{2\}  $ & $\{2,3\}$ & $\{2\}$ & $\{2\}$ \\
	$      ^\ast 3$ & $\{3\}  $ & $\{3\}  $ & $\{3\}$ & $\{3\}$
\end{tabular}
\end{center}
\subsection{Item d}
\begin{center}
\begin{tabular}{r | c c c c}
	$             \delta_N$ & $a      $ & $b      $ & $c    $ & $d    $ \\ \hline
	$\rightarrow \{1    \}$ & $\{1,2\}$ & $\{1\}  $ & $\{1\}$ & $\{1\}$ \\
	$            \{1,2  \}$ & $\{1,2\}$ & $\{1,2,3\}$ & $\{1,2\}$ & $\{1,2\}$ \\
	$      ^\ast \{1,2,3\}$ & $\{1,2,3\}$ & $\{1,2,3\}$ & $\{1,2,3\}$ & $\{1,2,3\}$ \\
\end{tabular}
\end{center}
\begin{center} \includegraphics[scale=0.12]{PA03_d} \end{center}
}