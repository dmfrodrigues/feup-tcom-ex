\setcounter{section}{3}
\section{Challenge Activity 4 - \texorpdfstring{$\varepsilon$}{epsilon}-NFAs}
\subsection{Exercise 1}
The NFA equivalent to a given $\varepsilon$-NFA is given by
\begin{alignat*}{2}
	NFA &= (\Sigma, Q_N, \delta_N, q_N, F_N)\\
	Q_N &\subseteq \mathscr{P}(Q_E)\\
	q_N &= \varepsilon close(q_E)\\
	F_N &= \{\varepsilon close(q)\colon q \in Q_E \wedge \varepsilon close(q) \cap F_E \neq \emptyset\}\\
	T(S,a)&= \bigcup_{q\in S}{\delta_E(q,a)}\\
	\delta_N(S,a) &= \{\varepsilon close(q)\colon q \in T(S,a)\}
\end{alignat*}
where $T(S,A)$ are states reachable from any of the members of $S$ through transitions with $a\in \Sigma$.\\
Informally, this corresponds to replacing every occurence of a state by its $\varepsilon$-closure, given all states of $\varepsilon close(q)$ can be reached the same way $q$ can.
\subsection{Exercise 2}
\begin{alignat*}{2}
    NFA &= (\Sigma, Q_N, \delta_N, q_N, F_N)\\
    \Sigma &= \{a,b\}\\
	q_N &= \varepsilon close(0)\\
	    &= \{0,1,2,4,7\}\\
	F_N &= \{\varepsilon close(q)\colon q \in Q_E \wedge \varepsilon close(q) \cap F_E \neq \emptyset\}\\
	    &= \{\{10\}\}
\end{alignat*}
\begin{alignat*}{2}
	\varepsilon close(0)&=\{0,1,2,4,7\} \\
	\varepsilon close(1)&=\{1,2,4\} \\
	\varepsilon close(2)&=\{2\} \\
	\varepsilon close(3)&=\{1,2,3,4,6,7\} \\
	\varepsilon close(4)&=\{4\} \\
	\varepsilon close(5)&=\{1,2,4,5,6,7\} \\
	\varepsilon close(6)&=\{1,2,4,7\} \\
	\varepsilon close(7)&=\{7\} \\
	\varepsilon close(8)&=\{8\} \\
	\varepsilon close(9)&=\{9\} \\
	\varepsilon close(10)&=\{10\}
\end{alignat*}
\begin{center}
\begin{tabular}{ r | c c }
 $\delta_N$ & $a$ & $b$ \\ \hline
 $\rightarrow \{0,1,2,  4,    7   \}$ & $\{\{1,2,3,4,6,7\},\{8\}\}$ & $\{\{1,2,4,5,6,7\}\}$\\%eclose(0)
 $            \{  1,2,3,4,  6,7   \}$ & $\{\{1,2,3,4,6,7\},\{8\}\}$ & $\{\{1,2,4,5,6,7\}\}$\\%eclose(3)
 $            \{  1,2,  4,5,6,7   \}$ & $\{\{1,2,3,4,6,7\},\{8\}\}$ & $\{\{1,2,4,5,6,7\}\}$\\%eclose(5)
 $            \{                8 \}$ & $\emptyset                $ & $\{\{9\}\}          $\\%eclose(8)
 $            \{                9 \}$ & $\emptyset                $ & $\{\{10\}\}         $\\%eclose(9)
 $        \ast\{                10\}$ & $\emptyset                $ & $\emptyset          $  %eclose(10)
\end{tabular}
\end{center}
Replacing sets by the $\varepsilon$-closures they correspond to, we have
\begin{center}
\begin{tabular}{ r | c c }
 $\delta_N$ & $a$ & $b$ \\ \hline
 $\rightarrow  0 $ & $\{3,8\}  $ & $\{5\}    $\\%eclose(0)
 $             3 $ & $\{3,8\}  $ & $\{5\}    $\\%eclose(3)
 $             5 $ & $\{3,8\}  $ & $\{5\}    $\\%eclose(5)
 $             8 $ & $\emptyset$ & $\{9\}    $\\%eclose(8)
 $             9 $ & $\emptyset$ & $\{10\}   $\\%eclose(9)
 $        \ast 10$ & $\emptyset$ & $\emptyset$  %eclose(10)
\end{tabular}
\end{center}
\begin{center}
	\begin{tikzpicture}[->,>=stealth',node distance=2.5cm,initial text=$ $,]
		\node[state, initial] (0) {$0$};
		\node[state, right of=0] (3) {$3$};
		\node[state, above right of=3] (5) {$5$};
		\node[state, below right of=3] (8) {$8$};
		\node[state, right of=8] (9) {$9$};
		\node[state, accepting, right of=9] (10) {$10$};

		\draw   (0)	edge[above                 	] node{$a$} (3)
				(0)	edge[below                 	] node{$a$} (8)
				(0)	edge[above, bend left=24   	] node{$b$} (5)

				(3)	edge[loop above            	] node{$a$} (3)
				(3)	edge[above	            	] node{$a$} (8)

				(3)	edge[right, bend right=10  	] node{$b$} (5)
				(5)	edge[above, bend right=10  	] node{$a$} (3)

				(5)	edge[loop above            	] node{$b$} (5)
				(5)	edge[right		           	] node{$a$} (8)

				(8)	edge[above		           	] node{$b$} (9)
				(9)	edge[above		           	] node{$b$} (10)
		;
	\end{tikzpicture}
\end{center}
Both the provided $\varepsilon$-NFA and the constructed NFA accept the same strings: those ended in $abb$.
\subsection{Exercise 3}
I think my scheme can be applied to every $\varepsilon$-NFA.
