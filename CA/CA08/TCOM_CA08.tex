\documentclass[docid=CA08]{tcom_CA}
\begin{document}
\setcounter{chapter}{7}
\exam{CFGs and Pushdown Automata}
\question{Exercise 1}
{

\questionitem{Item a}
\begin{theorem}
If $C$ is an ambiguous CFG, the equivalent PDA accepting by empty stack is non-deterministic.
\end{theorem}
\begin{proof}
A CFG is ambiguous iff there exists a string $w$ with at least two different left-most derivations, which is the same as stating there exists a string $w$ that, at some point of the left-most derivation, can derive into two different states through different productions for the same stack symbol. Let us say without loss of generality that production is of the form $A \rightarrow B\mid C$.\\
A PDA is non-deterministic iff there exists at least one triple $(q,a,X)$ such that $\#\delta(q, a, X) > 1$.\\
The conversion method to a PDA accepting by empty stack will give rise to the following transition: $\delta(q_0, \varepsilon, A)=\{(q_0,B),(q_0,C)\}$, which originates two different contents for the stack. The theorem is thus proven.
\end{proof}
\questionitem{Item b}
Consider the CFG defined by the production rule $A \rightarrow a$. The conversion to a PDA accepting by empty stack will yield the transition function
\begin{alignat*}{2}
	&\delta(q_0,\varepsilon,A) &&= \{(q_0, a)\}\\
	&\delta(q_0,a          ,a) &&= \{(q_0, \varepsilon)\} 
\end{alignat*}
that trivially describes a deterministic PDA.\\
If we consider \textit{non-deterministic} PDA in the strict sense that is is not trivial to convert to a deterministic DFA, then the statement is \textbf{false}.\\
If we however consider that all deterministic PDAs are particular cases of non-deterministic PDAs, then the statement if \textbf{true}.
}
\end{document}
