\documentclass[docid=TP01]{tcom_TP}
\begin{document}
\setcounter{chapter}{0}
\exam{TP01 - Proof by induction}
\question{Exercise 1}
\begin{theorem}
	\begin{equation*}
		\forall n\in\mathbb{N},\sum_{i=0}^{n}{(2i+1)}=(n+1)^2
	\end{equation*}
\end{theorem}
\begin{proof}
	Base case: $n=0$.
\begin{alignat*}{2}
	\sum_{i=0}^{n}{(2i+1)}=(n+1)^2
	&\iff \sum_{i=0}^{0}{(2i+1)}&&=(0+1)^2 \\
    &\iff (2*0+1)               &&=1^2 \\
    &\iff 1                     &&=1
\end{alignat*}
Inductive hypothesis:
\begin{equation*}
	\sum_{i=0}^{n}{(2i+1)}=(n+1)^2
\end{equation*}
Inductive step:
\begin{alignat*}{2}
	\sum_{i=0}^{n+1}{(2i+1)}=(n+2)^2
	&\iff \sum_{i=0}^{n}{(2i+1)}+[2(n+1)+1]&&=(n+2)^2 \\
    &\iff (n+1)^2+2n+3                     &&=(n+2)^2 \\
    &\iff n^2+2n+1+2n+3                    &&=n^2+4n+4 \\
    &\iff n^2+4n+4                         &&=n^2+4n+4
\end{alignat*}
Having proven the theorem holds for the base case, and the inductive step, we have proven the theorem.
\end{proof}
\pagebreak
\question{Exercise 2}
\begin{definition}[Set of all trees]
	$\mathbb{T}$ is the set of all trees as constructed by (1), (2) and (3) of the definition of tree.
\end{definition}
\begin{definition}[Operator $+S$]
\label{def:tplus}
	If $S\subseteq\mathbb{T}$ is a finite, non-empty set of trees, then $+S$ is the tree constructed through (2).
\end{definition}
\begin{remark}
	Definition \ref{def:tplus} also implies that for all trees $T$ that are not a single node, there is a finite set $S\subseteq\mathbb{T}$ such that $T=+S$.
\end{remark}
\begin{definition}[\#nodes of a tree]
	May $N\colon\mathbb{T}\rightarrow\mathbb{N}$ be such that
	\begin{alignat*}{2}
		N(T)= \begin{cases}
			  1                          & \text{if $T$ is a single node}\\
			  \sum_{T' \in S}{N(T')} + 1 & \text{where $S\subseteq\mathbb{T}$ is such that $+S=T$} 
			  \end{cases}
	\end{alignat*}
\end{definition}
\begin{definition}[\#edges of a tree]
	May $E\colon\mathbb{T}\rightarrow\mathbb{N}$ be such that
	\begin{alignat*}{2}
		E(T)= \begin{cases}
			  0                            & \text{if $T$ is a single node}\\
			  \sum_{T' \in S}{E(T')} + \#S & \text{where $S\subseteq\mathbb{T}$ is such that $+S=T$} 
			  \end{cases}
	\end{alignat*}
\end{definition}
\begin{theorem}
\begin{equation*}
\forall T \in \mathbb{T}, N(T)=E(T)+1
\end{equation*}
\end{theorem}
\begin{proof}
Base base: $T$ is a single node, which implies by (1) that $T\in\mathbb{T}$.
\begin{equation*}
N(T)=E(T)+1 \iff 1=0+1
\end{equation*}
Inductive hypothesis: $T_1,T_2,\ldots,T_K$ are trees that verify
\begin{equation*}
N(T_i)=E(T_i)+1 \iff N(T_i)-E(T_i)=1
\end{equation*}
Inductive step:
\begin{alignat*}{2}
S&\vcentcolon&&=\bigcup_{i=1}^{K}{\{T_i\}}\\
T& &&=+S
\end{alignat*}
\begin{alignat*}{2}
N(T)=E(T)+1 &\iff N(+S)                          =E(+S)+1 \\
            &\iff \sum_{i=1}^{K}{N(T_i)}+1       =K+\sum_{i=1}^{K}{E(T_i)} \\
            &\iff \sum_{i=1}^{K}{[N(T_i)-E(T_i)]}=K-1 \\
            &\iff \sum_{i=1}^{K}{1}              =K-1 \\
            &\iff K-1                            =K-1
\end{alignat*}
Having proven the theorem holds for the base case, and the inductive step, we have proven the theorem.
\end{proof}
\pagebreak
\question{Exercise 3}
\begin{definition}[Palindrome]
\label{def:palindrome}
$\alpha\in\Sigma^*$ is a palindrome iff:
\begin{enumerate}
  \item \label{itm:pal:1} $\alpha\in\Sigma^1$.
  \item \label{itm:pal:2} $\alpha=a+a, a\in\Sigma$.
  \item \label{itm:pal:3} $\alpha=a+\beta+a$ where $a\in\Sigma$ and $\beta$ is a palindrome.
\end{enumerate}
\end{definition}
\begin{theorem}
Every pal is a palindrome.
\end{theorem}
\begin{proof}
Base case: $\alpha\in\Sigma^1$.\\
By (1) of the definition of pal, $\alpha$ is a pal. Also by \eqref{itm:pal:1} of definition \ref{def:palindrome}, $\alpha$ is a palindrome.\\
Inductive hypothesis: $\beta$ is a pal and a palindrome.\\
Inductive step:\\
Let $\alpha=a+\beta+a, a\in\Sigma$. Given $\beta$ is a pal and $\alpha$ follows (2) of the definition of pal, $\alpha$ is a pal. Given $\beta$ is also a palindrome and $\alpha$ follows \eqref{itm:pal:2} of definition \ref{def:palindrome}, $\alpha$ is also a palindrome. Thus, $\alpha$ is a pal and a palindrome.\\
Having proven the theorem holds for the base case, and the inductive step, we have proven the theorem.
\end{proof}
\begin{definition}[Length of string]
\label{def:length}
	May $L\colon\Sigma^*\rightarrow\mathbb{N}_0$ be such that
	\begin{equation*}
		\forall\alpha\in\Sigma^*, \alpha\in\Sigma^{L(\alpha)}
	\end{equation*}
\end{definition}
\begin{enumerate}[label=Property \arabic*.,itemindent=*]
	\item $a\in\Sigma \implies L(a)=1$
	\item $L(\alpha+\beta)=L(\alpha)+L(\beta)$
\end{enumerate}
\begin{lemma}
\label{lem:palodd}
	For every pal $\alpha$, $L(\alpha) \equiv 1 \pmod{2}$
\end{lemma}
\begin{proof}
Base case: $\alpha\in\Sigma^1$.\\
By definition \ref{def:length}, $L(\alpha)=1$, and as such verifies $L(\alpha) \equiv 1 \pmod{2}$.\\
Inductive hypothesis: $\beta$ is a pal and $L(\beta) \equiv 1 \pmod{2}$.\\
Inductive step:\\
Let $\alpha=a+\beta+a, a\in\Sigma$. Given $\beta$ is a pal and $\alpha$ follows (2) of the definition of pal, $\alpha$ is a pal.
\begin{equation*}
	L(\alpha)=L(a)+L(\beta)+L(a)=L(\beta)+2
\end{equation*}
\begin{alignat*}{2}
	L(\alpha) \equiv 1 \pmod{2} &\iff L(\beta)+2 &&\equiv 1 \pmod{2} \\
		                        &\iff L(\beta)   &&\equiv 1 \pmod{2}
\end{alignat*}
Having proven the lemma holds for the base case, and the inductive step, we have proven the lemma correct.
\end{proof}
\begin{theorem}
	There exists at least one palindrome that is not a pal.
\end{theorem}
\begin{proof}
	Consider any symbol $a\in\Sigma$. A string $\alpha=a+a$ is a palindrome by \eqref{itm:pal:2} of definition \ref{def:palindrome}. However, $L(\alpha)=2$ and as such does not verify $\alpha$, $L(\alpha) \equiv 1 \pmod{2}$. By lemma \eqref{lem:palodd}, $\alpha$ can not be a pal.
\end{proof}
\pagebreak
\question{Exercise 4}
\begin{theorem}
	\begin{equation*}
		\forall n \in \mathbb{N}\cap[4,+\infty),\exists a,b\in\mathbb{N}\colon 10n=20a+50b
	\end{equation*}
\end{theorem}
\begin{proof}
	\begin{alignat*}{4}
		&\forall n \in \mathbb{N}\cap[4,+\infty),\exists a,b\in\mathbb{N}\colon 10&&n=&&20a+&&50b \iff \\
		&\forall n \in \mathbb{N}\cap[4,+\infty),\exists a,b\in\mathbb{N}\colon   &&n=&&2a +&&5b
	\end{alignat*}
	The structure over which the proof will be done is the naturals $\mathbb{N}$ greater or equal to $4$.\\
	Base case: $n=4$.
	\begin{equation*}
		(a,b)=(2,0) \implies 2a+5b=4
	\end{equation*}
	Base case: $n=5$.
	\begin{equation*}
		(a,b)=(0,1) \implies 2a+5b=5
	\end{equation*}
	Inductive hypothesis: $\exists a,b\in\mathbb{N}\colon n=2a+5b$\\
	Inductive step: if $n$ can be decomposed into $n=2a+5b$, then $n'=n+2$ can be decomposed into $n'=2a'+5b'$ where $(a',b')=(a+1,b)$.\\
	The inductive step means that, if we can decompose a number $n$, all numbers greater than $n$ with the same parity can also be decomposed.\\
	Since we can decompose $n=4$ and $n=5$, we can decompose all numbers greater or equal to $4$ with both parities, thus the theorem is proven correct.
\end{proof}
\question{Exercise 5}
\begin{theorem}
	\begin{equation*}
		\forall n \in \mathbb{N}\cap[2,+\infty),\prod_{i=2}^{n}{\left(1-\frac{1}{i}\right)}=\frac{1}{n}
	\end{equation*}
\end{theorem}
\begin{proof}
	Base case: $n=2$.
	\begin{alignat*}{2}
		\prod_{i=2}^{n}{\left(1-\frac{1}{i}\right)}=\frac{1}{n}
		&\iff \prod_{i=2}^{2}{\left(1-\frac{1}{i}\right)}&&=\frac{1}{2} \\
		&\iff 1-\frac{1}{2}                              &&=\frac{1}{2}
	\end{alignat*}
	Inductive hypothesis:
	\begin{equation*}
		\prod_{i=2}^{n}{\left(1-\frac{1}{i}\right)}=\frac{1}{n}
	\end{equation*}
	Inductive step:
	\begin{alignat*}{2}
		\prod_{i=2}^{n+1}{\left(1-\frac{1}{i}\right)}=\frac{1}{n+1}
		&\iff \left(1-\frac{1}{n+1}\right)\prod_{i=2}^{n}{\left(1-\frac{1}{i}\right)}&&=\frac{1}{n+1} \\
		&\iff \left(1-\frac{1}{n+1}\right)\frac{1}{n}                                &&=\frac{1}{n+1} \\
		&\iff \frac{n+1}{n+1}-\frac{1}{n+1}                                          &&=\frac{n}{n+1} \\
		&\iff \frac{n}{n+1}                                                          &&=\frac{n}{n+1}
	\end{alignat*}
	Having proven the theorem holds for the base case, and the inductive step, we have proven the theorem.
\end{proof}
\pagebreak
\question{Exercise 6}
\begin{theorem}
	\begin{equation*}
		\forall n \in \mathbb{N}\cap[1,+\infty),n^2+n \equiv 0 \pmod{2}
	\end{equation*}
\end{theorem}
\begin{proof}
	Base case: $n=1$.
	\begin{alignat*}{2}
		n^2+n \equiv 0 \pmod{2}
		&\iff 1^2+1 &&\equiv 0 \pmod{2} \\
		&\iff 2     &&\equiv 0 \pmod{2}
	\end{alignat*}
	Inductive hypothesis:
	\begin{equation*}
		n^2+n \equiv 0 \pmod{2}
	\end{equation*}
	Inductive step:
	\begin{alignat*}{2}
			(n+1)^2+(n+1) \equiv 0 \pmod{2}
			&\iff (n^2+2n+1)+(n+1) &&\equiv 0 \pmod{2} \\
			&\iff n^2+3n+2         &&\equiv 0 \pmod{2} \\
			&\iff n^2+n            &&\equiv 0 \pmod{2}
	\end{alignat*}
	Having proven the theorem holds for the base case, and the inductive step, we have proven the theorem.
\end{proof}
\question{Exercise 7}
\begin{lemma}
	\begin{equation*}
		\forall n \in \mathbb{N},\sum_{i=0}^{n}{i^3}={\left[\frac{n(n+1)}{2}\right]}^2
	\end{equation*}
\end{lemma}
\begin{proof}
	Base case: $n=0$.
	\begin{alignat*}{2}
		\sum_{i=0}^{n}{i^3}={\left[\frac{n(n+1)}{2}\right]}^2
		&\iff \sum_{i=0}^{0}{i^3} && ={\left[\frac{0(0+1)}{2}\right]}^2 \\
		&\iff 0^3 && =0
	\end{alignat*}
	Inductive hypothesis:
	\begin{equation*}
		\sum_{i=0}^{n}{i^3}={\left[\frac{n(n+1)}{2}\right]}^2
	\end{equation*}
	Inductive step:
	\begin{alignat*}{3}
		\sum_{i=0}^{n+1}{i^3}={\left[\frac{(n+1)(n+2)}{2}\right]}^2
		&\iff \sum_{i=0}^{n}{i^3} +(n+1)^3                               &&= {\left[\frac{(n+1)(n+2)}{2}\right]}^2 \\
		&\iff {\left[\frac{n(n+1)}{2}\right]}^2 +(n+1)^3                 &&= {\left[\frac{(n+1)(n+2)}{2}\right]}^2 \\
		&\iff (n+1)^2 \left[ {\left(\frac{n}{2}\right)}^2 +(n+1) \right] &&= (n+1)^2{\left[\frac{n+2}{2}\right]}^2 \\
		&\iff {\left(\frac{n}{2}\right)}^2 +(n+1)                        &&= {\left[\frac{n+2}{2}\right]}^2 \\
		&\iff \frac{n^2}{4}+n+1                                          &&= \frac{n^2+4n+4}{4} \\
		&\iff \frac{n^2}{4}+n+1                                          &&= \frac{n^2}{4}+n+1
	\end{alignat*}
	Having proven the theorem holds for the base case, and the inductive step, we have proven the theorem.
\end{proof}
\pagebreak
\question{Exercise 8}
\begin{definition}[Binary tree]
\label{def:bt}
	$B$ is a binary tree iff:
	\begin{enumerate}
  		\item \label{itm:bt:1} It is constitued of $1$ root.
  		\item \label{itm:bt:2} It is constitued of $1$ root connected by $1$ edge to  subtree $B_1$.
  		\item \label{itm:bt:3} it is constitued of $1$ root connected by $2$ edges to subtrees $B_1,B_2$.
	\end{enumerate}
\end{definition}
\begin{remark}
	Definition \ref{def:bt} is just a different way to express the exercise statement.
\end{remark}
\begin{definition}[Set of all binary trees]
	$\mathbb{B}$ is the set of all binary trees.
\end{definition}
\begin{definition}[\#fullnodes of a binary tree]
	May $N\colon\mathbb{B}\rightarrow\mathbb{N}$ be such that
	\begin{alignat*}{2}
		N(B)= \begin{cases}
			  0               & \text{in case \ref{itm:bt:1}}\\
			  N(B_1)          & \text{in case \ref{itm:bt:2}}\\
			  1+N(B_1)+N(B_2) & \text{in case \ref{itm:bt:3}}\\
			  \end{cases}
	\end{alignat*}
\end{definition}
\begin{definition}[\#leaves of a binary tree]
	May $L\colon\mathbb{B}\rightarrow\mathbb{N}$ be such that
	\begin{alignat*}{2}
		L(B)= \begin{cases}
			  1             & \text{in case \ref{itm:bt:1}}\\
			  L(B_1)        & \text{in case \ref{itm:bt:2}}\\
			  L(B_1)+L(B_2) & \text{in case \ref{itm:bt:3}}\\
			  \end{cases}
	\end{alignat*}
\end{definition}
\begin{theorem}
	\begin{equation*}
		\forall B \in \mathbb{B},L(B)=N(B)+1
	\end{equation*}
\end{theorem}
\begin{proof}
	Base case: case \ref{itm:bt:1}.
	\begin{equation*}
		L(B)=N(B)+1 \iff 1=0+1
	\end{equation*}
	Inductive hypothesis: $B_1$, $B_2$ are binary trees that verify
	\begin{gather*}
		L(B_1)=N(B_1)+1\\
		L(B_2)=N(B_2)+1
	\end{gather*}
	Inductive step:
	\begin{itemize}[label={}]
		\item Case \ref{itm:bt:2}: Let $B$ be a binary tree connected to $B_1$.
			\begin{alignat*}{2}
				L(B)=N(B)+1
				&\iff L(B_1)=N(B_1)+1
			\end{alignat*}
		\item Case \ref{itm:bt:3}: Let $B$ be a binary tree connected to $B_1$,$B_2$.
			\begin{alignat*}{3}
				\begin{cases}
					L(B_1)=N(B_1)+1\\
					L(B_2)=N(B_2)+1
				\end{cases}
				&\implies &&L(B_1)+L(B_2)&&=N(B_1)+1+N(B_2)+1 \\
				&\iff     &&L(B)         &&=N(B)+1
			\end{alignat*}
	\end{itemize}
	Having proven the theorem holds for the base case, and the inductive step for both cases \ref{itm:bt:2} and \ref{itm:bt:3}, we have proven the theorem.
\end{proof}
\end{document}
