\documentclass[docid=TP09]{tcom_TP}
\begin{document}
\setcounter{section}{8}
\section{TP09 - Pushdown Automata (PDAs)}
{
\renewcommand{\thesubsubsection}{\thesubsection\alph{subsubsection}}
\subsection{Exercise 1}
\subsubsection{Item a}
\begin{alignat*}{2}
	S \rightarrow aSb\mid aSbb\mid \varepsilon
\end{alignat*}
\subsubsection{Item b} \label{TP09_1b}
PDA accepting by empty stack.
\begin{center}
\begin{minipage}[c]{0.30\textwidth}
\begin{alignat*}{2}
	PDA    &= (Q, \Sigma, \Gamma, \delta, q_0, Z_0)\\
	Q      &= \{q_0\}\\
	\Sigma &= \{a,b\}\\
	\Gamma &= \{a,b,S\}\\
	Z_0    &= S
\end{alignat*}
\end{minipage}%
\begin{minipage}[c]{0.65\textwidth}
\begin{alignat*}{2}
	\delta \colon & Q \times (\Sigma \cup \{\varepsilon\}) \times (\Gamma \cup \{\varepsilon\}) &&\rightarrow \mathscr{P}(Q \times \Gamma^*))\\
	&\delta(q_0, \varepsilon, S) &&=\{(q_0, aSb), (q_0, aSbb), (q_0, \varepsilon)\}\\
	&\delta(q_0, a, a) &&= \{(q_0, \varepsilon)\}\\
	&\delta(q_0, b, b) &&= \{(q_0, \varepsilon)\}
\end{alignat*}
\end{minipage}
\end{center}
\subsubsection{Item c}
\begin{center}
	\begin{tikzpicture}[->,>=stealth',node distance=1.7cm,initial text=$ $,state/.style={circle,inner sep=2pt}]
		\footnotesize
		\node[align=center](10) at (+0.00,+0.0) {$q_0$,\\aabbb,\\S};
		\node[align=center](11) at (+2.0,+0.0) {$q_0$,\\aabbb,\\$\varepsilon$};
		\node[align=center](20) at (-1.0,-1.5) {$q_0$,\\aabbb,\\aSb};
		\node[align=center](21) at (+1.0,-1.5) {$q_0$,\\aabbb,\\aSbb};
		\node[align=center](30) at (-3.0,-3.0) {$q_0$,\\abbb, \\b};
		\node[align=center](31) at (-1.0,-3.0) {$q_0$,\\abbb, \\Sb};
		\node[align=center](32) at (+1.0,-3.0) {$q_0$,\\abbb, \\Sbb};
		\node[align=center](33) at (+3.0,-3.0) {$q_0$,\\abbb, \\bb};
		\node[align=center](40) at (-2.0,-4.5) {$q_0$,\\abbb, \\aSbb};
		\node[align=center](41) at (+0.0,-4.5) {$q_0$,\\abbb, \\aSbbb};
		\node[align=center](42) at (+2.0,-4.5) {$q_0$,\\abbb, \\aSbbbb};
		\node[align=center](50) at (-6.0,-6.0) {$q_0$,\\bb, \\b};
		\node[align=center](51) at (-4.0,-6.0) {$q_0$,\\bbb, \\bb};
		\node[align=center](52) at (-2.0,-6.0) {$q_0$,\\bbb, \\Sbb};
		\node[align=center](53) at (+0.0,-6.0) {$q_0$,\\bbb, \\Sbbb};
		\node[align=center](54) at (+2.0,-6.0) {$q_0$,\\bbb, \\Sbbbb};
		\node[align=center](55) at (+4.0,-6.0) {$q_0$,\\bbb, \\bbbb};
		\node[align=center](56) at (+6.0,-6.0) {$q_0$,\\bb, \\bbb};
		\node[align=center](60) at (-6.0,-7.5) {$q_0$,\\b, \\$\varepsilon$};
		\node[align=center](61) at (-4.0,-7.5) {$q_0$,\\bbb, \\aSbbb};
		\node[align=center](62) at (-2.0,-7.5) {$q_0$,\\bbb, \\aSbbbb};
		\node[align=center](63) at (+0.0,-7.5) {$q_0$,\\bbb, \\bbb};
		\node[align=center](64) at (+2.0,-7.5) {$q_0$,\\bbb, \\aSbbbbb};
		\node[align=center](65) at (+4.0,-7.5) {$q_0$,\\bbb, \\aSbbbbbb};
		\node[align=center](66) at (+6.0,-7.5) {$q_0$,\\b, \\bb};
		\node[align=center](70) at (+0.0,-9.0) {$q_0$,\\bb, \\bb};
		\node[align=center](71) at (+6.0,-9.0) {$q_0$,\\$\varepsilon$, \\b};
		\node[align=center](80) at (+0.0,-10.5){$q_0$,\\b, \\b};
		\node[align=center](90) at (+0.0,-12.0){$q_0$,\\$\varepsilon$, \\$\varepsilon$};

		\draw   (10)	edge (11)
				(10)	edge (20)
				(10)	edge (21)
				(20)	edge (31)
				(21)	edge (32)
				(31)	edge (30)
				(32)	edge (33)
				(31)	edge (40)
				(31)	edge (41)
				(32)	edge (41)
				(32)	edge (42)
				(51)	edge (50)
				(52)	edge (51)
				(40)	edge (52)
				(41)	edge (53)
				(42)	edge (54)
				(54)	edge (55)
				(55)	edge (56)
				(50)	edge (60)
				(52)	edge (61)
				(52)	edge (62)
				(53)	edge (62)
				(53)	edge (63)
				(53)	edge (64)
				(54)	edge (64)
				(54)	edge (65)
				(56)	edge (66)
				(63)	edge (70)
				(66)	edge (71)
				(70)	edge (80)
				(80)	edge (90)
				;
	\end{tikzpicture}
\end{center}
\pagebreak
\subsubsection{Item d}
When the input string is $aaabb$, the stack has always more $b$'s than can be consumed from the remanescent input.
\begin{center}
	\begin{tikzpicture}[->,>=stealth',node distance=1.7cm,initial text=$ $,state/.style={circle,inner sep=2pt}]
		\footnotesize
		\node[align=center](10) at (+0.0,+0.0) {$q_0$,\\aaabb,\\S};
		\node[align=center](11) at (+3.0,-1.5) {$q_0$,\\aaabb,\\$\varepsilon$};
		\node[align=center](20) at (-1.0,-1.5) {$q_0$,\\aaabb,\\aSb};
		\node[align=center](21) at (+1.0,-1.5) {$q_0$,\\aaabb,\\aSbb};
		\node[align=center](31) at (-1.0,-3.0) {$q_0$,\\aabb, \\Sb};
		\node[align=center](32) at (+1.0,-3.0) {$q_0$,\\aabb, \\Sbb};
		\node[align=center](30) at (-4.0,-4.5) {$q_0$,\\aabb, \\b};
		\node[align=center](33) at (+4.0,-4.5) {$q_0$,\\aabb, \\bb};
		\node[align=center](40) at (-2.0,-4.5) {$q_0$,\\aabb, \\aSbb};
		\node[align=center](41) at (+0.0,-4.5) {$q_0$,\\aabb, \\aSbbb};
		\node[align=center](42) at (+2.0,-4.5) {$q_0$,\\aabb, \\aSbbbb};
		\node[align=center](52) at (-2.0,-6.0) {$q_0$,\\abb, \\Sbb};
		\node[align=center](53) at (+0.0,-6.0) {$q_0$,\\abb, \\Sbbb};
		\node[align=center](54) at (+2.0,-6.0) {$q_0$,\\abb, \\Sbbbb};
		\node[align=center](51) at (-6.0,-7.5) {$q_0$,\\abb, \\bb};
		\node[align=center](55) at (+6.0,-7.5) {$q_0$,\\abb, \\bbbb};
		\node[align=center](61) at (-4.0,-7.5) {$q_0$,\\abb, \\aSbbb};
		\node[align=center](62) at (-2.0,-7.5) {$q_0$,\\abb, \\aSbbbb};
		\node[align=center](63) at (+0.0,-7.5) {$q_0$,\\abb, \\bbb};
		\node[align=center](64) at (+2.0,-7.5) {$q_0$,\\abb, \\aSbbbbb};
		\node[align=center](65) at (+4.0,-7.5) {$q_0$,\\abb, \\aSbbbbbb};
		\node[align=center](70) at (-4.0,-9.0) {$q_0$,\\bb, \\Sbbb};
		\node[align=center](71) at (-2.0,-9.0) {$q_0$,\\bb, \\Sbbbb};
		\node[align=center](72) at (+2.0,-9.0) {$q_0$,\\bb, \\Sbbbbb};
		\node[align=center](73) at (+4.0,-9.0) {$q_0$,\\bb, \\Sbbbbbb};
		\node[align=center](8a) at (-7.5,-10.5) {$q_0$,\\bb, \\bbb};
		\node[align=center](80) at (-6.0,-10.5) {$q_0$,\\bb, \\aSbbbb};
		\node[align=center](81) at (-4.0,-10.5) {$q_0$,\\bb, \\aSbbbbb};
		\node[align=center](82) at (-2.0,-10.5) {$q_0$,\\bb, \\bbbbb};
		\node[align=center](83) at (+0.0,-10.5) {$q_0$,\\bb, \\aSbbbbbb};
		\node[align=center](84) at (+2.0,-10.5) {$q_0$,\\bb, \\bbbbb};
		\node[align=center](85) at (+4.0,-10.5) {$q_0$,\\bb, \\aSbbbbbbb};
		\node[align=center](86) at (+6.0,-10.5) {$q_0$,\\bb, \\aSbbbbbbbb};
		\node[align=center](8z) at (+7.5,-10.5) {$q_0$,\\bb, \\bbbbbb};
		\node[align=center](90) at (-7.5,-12.0) {$q_0$,\\b, \\bb};
		\node[align=center](91) at (-2.0,-12.0) {$q_0$,\\b, \\bbb};
		\node[align=center](92) at (+2.0,-12.0) {$q_0$,\\b, \\bbbb};
		\node[align=center](93) at (+7.5,-12.0) {$q_0$,\\b, \\bbbbb};
		\node[align=center](00) at (-7.5,-13.5) {$q_0$,\\$\varepsilon$, \\b};
		\node[align=center](01) at (-2.0,-13.5) {$q_0$,\\$\varepsilon$, \\bb};
		\node[align=center](02) at (+2.0,-13.5) {$q_0$,\\$\varepsilon$, \\bbb};
		\node[align=center](03) at (+7.5,-13.5) {$q_0$,\\$\varepsilon$, \\bbbb};
		

		\draw   (10)	edge (11)
				(10)	edge (20)
				(10)	edge (21)
				(20)	edge (31)
				(21)	edge (32)
				(31)	edge (30)
				(32)	edge (33)
				(31)	edge (40)
				(31)	edge (41)
				(32)	edge (41)
				(32)	edge (42)
				(52)	edge (51)
				(40)	edge (52)
				(41)	edge (53)
				(42)	edge (54)
				(54)	edge (55)
				(52)	edge (61)
				(52)	edge (62)
				(53)	edge (62)
				(53)	edge (63)
				(53)	edge (64)
				(54)	edge (64)
				(54)	edge (65)
				(61)	edge (70)
				(62)	edge (71)
				(64)	edge (72)
				(65)	edge (73)
				(70)	edge (8a)
				(70)	edge (80)
				(70)	edge (81)
				(71)	edge (81)
				(71)	edge (82)
				(71)	edge (83)
				(72)	edge (83)
				(72)	edge (84)
				(72)	edge (85)
				(73)	edge (85)
				(73)	edge (86)
				(73)	edge (8z)
				(8a)	edge (90)
				(82)	edge (91)
				(84)	edge (92)
				(8z)	edge (93)
				(90)	edge (00)
				(91)	edge (01)
				(92)	edge (02)
				(93)	edge (03)
				;
	\end{tikzpicture}
\end{center}
\subsection{Exercise 2} \label{TP09_2}
\begin{center}
\begin{minipage}[c]{0.30\textwidth}
\begin{alignat*}{2}
	PDA    &= (Q, \Sigma, \Gamma, \delta, q_0, Z_0)\\
	Q      &= \{q_0\}\\
	\Sigma &= \{0,1\}\\
	\Gamma &= \{0,1,S,A\}\\
	Z_0    &= S
\end{alignat*}
\end{minipage}%
\begin{minipage}[c]{0.60\textwidth}
\begin{alignat*}{2}
	\delta \colon Q \times (\Sigma \cup \{\varepsilon\}) \times \Gamma & \rightarrow \mathscr{P}(Q \times \Gamma^*)\\
	\delta(q_0, \varepsilon, S) &= \{(q_0, A1A)\} \\
	\delta(q_0, \varepsilon, A) &= \{(q_0,0A), (q_0, 1A),(q_0,\varepsilon)\}\\
	\delta(q_0, 0, 0) &= \{(q_0, \varepsilon)\}\\
	\delta(q_0, 1, 1) &= \{(q_0, \varepsilon\}
\end{alignat*}
\end{minipage}
\end{center}
\subsection{Exercise 3} \label{TP09_3}
\begin{center}
\begin{minipage}[c]{0.4\textwidth}
\begin{alignat*}{2}
	PDA ~ P &= (Q,\Sigma,\Gamma,\delta, q_0, Z_0)\\
	Q       &= \{q_0\}\\
	\Sigma  &= \{e,i\}\\
	\Gamma  &= \{e,i,S\}\\
	Z_0     &= S
\end{alignat*}
\end{minipage}%
\begin{minipage}[c]{0.6\textwidth}
\begin{alignat*}{2}
	\delta \colon Q \times (\Sigma \cup \{\varepsilon\}) \times \Gamma & \rightarrow \mathscr{P}(Q \times \Gamma ^*)\\
	\delta (q_0,\varepsilon ,S) &=\{(q_0,SS),(q_0,iS),(q_0,iSeS), (q_0, \varepsilon)\}\\
	\delta (q_0,e,e)&=\{(q_0,\varepsilon)\}\\
	\delta (q_0,i,i)&=\{(q_0,\varepsilon)\}
\end{alignat*}
\end{minipage}
\end{center}
\subsection{Exercise 4}
\subsubsection{Item a}
\begin{center}
	\begin{tikzpicture}[->,>=stealth',node distance=1.7cm,initial text=$ $,state/.style={circle,inner sep=2pt}]
		\node[				](1) {$(p, 1100, Z)$};
		\node[below of=1	](21) {$(p, 100, 1Z)$};
		\node[right of=21	](22) {$(s, 1100, \varepsilon)$};
		\node[below of=21	](3) {$(p, 00, 11Z)$};
		\node[below of=3	](4) {$(p, 0, 1Z)$};
		\node[below of=4	](5) {$(p, \varepsilon, Z)$};
		\node[below of=5	](6) {$(s, \varepsilon, \varepsilon)$};
		
		\draw   (1)		edge (21)
				(1)		edge (22)
				(21)	edge (3)
				(3)		edge (4)
				(4)		edge (5)
				(5)		edge (6)
				;
	\end{tikzpicture}
\end{center}
\subsubsection{Item b}
The string $1100$ is accepted, given this PDA accepts by empty stack, and using it as input the PDA reaches an instantaneous description with empty remanescent input and empty stack.
\subsubsection{Item c}
\begin{center}
	\begin{tikzpicture}[->,>=stealth',node distance=1.7cm,initial text=$ $,state/.style={circle,inner sep=2pt}]
		\node[				](1) {$(p, 101, Z)$};
		\node[below of=1	](21) {$(p, 01, 1Z)$};
		\node[right of=21	](22) {$(p, 101, \varepsilon)$};
		\node[below of=21	](3) {$(p, 1, Z)$};
		\node[below of=3	](41) {$(p, \varepsilon, 1Z)$};
		\node[right of=41	](42) {$(p, 1, \varepsilon)$};
		
		\draw   (1)		edge (21)
				(1)		edge (22)
				(21)	edge (3)
				(3)		edge (41)
				(3)		edge (42)
				;
	\end{tikzpicture}
\end{center}
\subsection{Exercise 5}
This statement is false, given a CFG is ambiguous iff there exists a string with at least two different syntax trees. Having two different syntax trees can be guaranteed by having two different rightmost derivations, or two different leftmost derivations. Having different leftmost and rightmost derivations is not enough to be ambiguous.
}
\end{document}
