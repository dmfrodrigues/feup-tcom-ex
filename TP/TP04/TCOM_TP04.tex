\documentclass[docid=TP04]{tcom_TP}
\begin{document}
\setcounter{section}{3}
\section{TP04 - \texorpdfstring{$\varepsilon$}{e}-NFAs}
{
\renewcommand{\thesubsubsection}{\thesubsection\alph{subsubsection}}
\subsection{Exercise 1}
\subsubsection{Item a}
\begin{center}
	\begin{tikzpicture}[->,>=stealth',node distance=1.5cm,initial text=$ $,]
		\footnotesize
		\node[state, initial] (0) {$0$};
		\node[state, above right of=0] (00) {$00$};
		\node[state, right of=00] (01) {$01$};
		\node[state, accepting, right of=01] (02) {$02$};
		\node[state, below right of=0] (10) {$10$};
		\node[state, right of=10] (11) {$11$};
		\node[state, right of=11] (12) {$12$};
		\node[state, accepting, right of=12] (13) {$13$};
		
		\draw   (0) 	edge[left		] node{$\varepsilon$} (00)
				(0) 	edge[left		] node{$\varepsilon$} (10)

				(00) 	edge[above		] node{$0$} (01)
				(01) 	edge[above		] node{$1$} (02)
				(02) 	edge[bend left=35, below	] node{$\varepsilon$} (00)

				(10) 	edge[below		] node{$0$} (11)
				(11) 	edge[below		] node{$1$} (12)
				(12) 	edge[below		] node{$0$} (13)
				(13) 	edge[bend right=35, above	] node{$\varepsilon$} (10)
				;
	\end{tikzpicture}
\end{center}
\subsubsection{Item b}
This $\varepsilon$-NFA is actually a regular NFA; suggestions for a simpler, working $\varepsilon$-NFA are welcome.
\begin{center}
	\begin{tikzpicture}[->,>=stealth',node distance=1.46cm,initial text=$ $,]
		\footnotesize
		\node[state, initial] (0) {$0$};
		\node[state, accepting, right of=0] (1)  {$1$};
		\node[state, accepting, right of=1] (2)  {$2$};
		\node[state, accepting, right of=2] (3)  {$3$};
		\node[state, accepting, right of=3] (4)  {$4$};
		\node[state, accepting, right of=4] (5)  {$5$};
		\node[state, accepting, right of=5] (6)  {$6$};
		\node[state, accepting, right of=6] (7)  {$7$};
		\node[state, accepting, right of=7] (8)  {$8$};
		\node[state, accepting, right of=8] (9)  {$9$};
		\node[state, accepting, right of=9] (10) {$10$};
	
		\draw   (0) 	edge[loop above		] node{$0,1$} (0)
				(0) 	edge[above			] node{$1$} (1)
				(1) 	edge[above			] node{$0,1$} (2)
				(2) 	edge[above			] node{$0,1$} (3)
				(3) 	edge[above			] node{$0,1$} (4)
				(4) 	edge[above			] node{$0,1$} (5)
				(5) 	edge[above			] node{$0,1$} (6)
				(6) 	edge[above			] node{$0,1$} (7)
				(7) 	edge[above			] node{$0,1$} (8)
				(8) 	edge[above			] node{$0,1$} (9)
				(9) 	edge[above			] node{$0,1$} (10)
				;
	\end{tikzpicture}
\end{center}
\subsection{Exercise 2}
\subsubsection{Item a}
\begin{alignat*}{2}
	\varepsilon close(p)&=\{p,q,r\}\\
	\varepsilon close(q)&=\{q\}\\
	\varepsilon close(r)&=\{r\}
\end{alignat*}
\subsubsection{Item b}
We will use the DFA from \ref{sssec:TP04_2_c} to answer to this item. The only reject state is $\emptyset$, which is also a dead state, so the only strings that are not accepted are the ones that end up locked in $\emptyset$. From that, it is easy to conclude that the only strings with length less or equal to $3$ that are rejected are $bba$ and $bbb$.
\subsubsection{Item c} \label{sssec:TP04_2_c}
\begin{center}
	\begin{tikzpicture}[->,>=stealth',node distance=2cm,initial text=$ $,]
		\footnotesize
		\node[state, initial, accepting] (pqr) {$\{p,q,r\}$};
		\node[state, accepting, above right of=pqr] (qr) {$\{q,r\}$};
		\node[state, accepting, below right of=qr] (r) {$\{r\}$};
		\node[state, right of=r] (0) {$\emptyset$};
	
		\draw   (pqr) 	edge[loop above				] node{$a, c$} (pqr)
				(pqr) 	edge[bend right=10, right	] node{$b$} (qr)

				(qr) 	edge[bend right=10, above	] node{$a, c$} (pqr)
				(qr) 	edge[right					] node{$b$} (r)

				(r) 	edge[above					] node{$a,b$} (0)
				(r) 	edge[below					] node{$c$} (pqr)

				(0) 	edge[loop above				] node{$a,b,c$} (0)
				;
	\end{tikzpicture}
\end{center}
\begin{center}
\begin{minipage}[c]{0.35\textwidth}
\begin{alignat*}{2}
	DFA       &= (\Sigma, Q_D, \delta_D, \{p,q,r\}, F_D)\\
	\Sigma    &= \{a,b\}\\
	Q_D       &= \{\{p,q,r\},\{q,r\},\{r\},\emptyset\}\\
	F_D       &= \{\{p,q,r\},\{q,r\},\{r\}\}\\
	\delta_D &\colon Q_D \times \Sigma \rightarrow Q_D
\end{alignat*}
\end{minipage}%
\begin{minipage}[c]{0.45\textwidth}
\begin{center}
\begin{tabular}{ r | c c c }
    $\delta_D$ & $a$ & $b$ & $c$ \\ \hline
    $\rightarrow^* \{p,q,r\}$ & $\{p,q,r\}$ & $\{  q,r\}$ & $\{p,q,r\}$ \\
    $           ^* \{  q,r\}$ & $\{p,q,r\}$ & $\{    r\}$ & $\{p,q,r\}$ \\
    $           ^* \{    r\}$ & $\emptyset$ & $\emptyset$ & $\{p,q,r\}$\\
    $ \emptyset                $ & $\emptyset$ & $\emptyset$ & $\emptyset$
\end{tabular}
\end{center}
\end{minipage}
\end{center}
\subsection{Exercise 3}
\subsubsection{Item a}
This $\varepsilon$-NFA is actually a regular NFA; suggestions for a simpler, working $\varepsilon$-NFA are welcome.
\begin{center}
	\begin{minipage}[c]{0.35\textwidth}
		\begin{alignat*}{2}
			\varepsilon NFA &= (\Sigma, Q_E, \delta_E, 0, F_E)\\
			\Sigma &= \{a,b\}\\
			Q_E    &= \{0,1,2\}\\
			F_E    &= \{0,1,2\}\\
			\delta_E &\colon Q_E \times \Sigma \rightarrow \mathscr{P}(Q_E)
		\end{alignat*}
	\end{minipage}
	\begin{minipage}[c]{0.35\textwidth}
		\begin{center}
		\begin{tabular}{ r | c c c }
 			$\delta_E          $ & $\varepsilon$ & $a    $ & $b    $ \\ \hline
 			$\rightarrow^* 0$ & $\emptyset$ & $\{1\}$ & $\{0\}$ \\  
 			$                 1$ & $\emptyset$ & $\{2\}$ & $\{0\}$ \\
 			$                 2$ & $\emptyset$ & $\emptyset$ & $\{0\}$
		\end{tabular}
		\end{center}
	\end{minipage}
	\begin{minipage}[c]{0.28\textwidth}
		\begin{center}
			\begin{tikzpicture}[->,>=stealth',node distance=1.5cm,initial text=$ $,]
				\footnotesize
				\node[state, initial, accepting] (0) {$0$};
				\node[state, below right of=0] (1) {$1$};
				\node[state, above right of=1] (2) {$2$};
				
				\draw   (0) 	edge[loop above				] node{$b$} (0)
						(0) 	edge[bend right=20, left	] node{$a$} (1)
						(1) 	edge[below					] node{$a$} (2)
						(1) 	edge[bend right=20, right	] node{$b$} (0)
						(2) 	edge[above					] node{$b$} (0)
						;
			\end{tikzpicture}
		\end{center}
	\end{minipage}
\end{center}
\subsubsection{Item b}
This $\varepsilon$-NFA is actually a regular NFA; suggestions for a simpler, working $\varepsilon$-NFA are welcome.
\begin{center}
	\begin{minipage}[c]{0.33\textwidth}
		\begin{alignat*}{2}
			\varepsilon NFA &= (\Sigma, Q_E, \delta_E, 0, F_E)\\
			\Sigma &= \{a,b,c\}\\
			Q_E    &= \{00,01,10,11\}\\
			F_E    &= \{00\}\\
			\delta_E &\colon Q_E \times \Sigma \rightarrow \mathscr{P}(Q_E)
		\end{alignat*}
	\end{minipage}
	\begin{minipage}[c]{0.38\textwidth}
		\begin{center}
		\begin{tabular}{ r | c c c c }
 			$\delta_E           $ & $\varepsilon$ & $a    $ & $b    $ & $c$ \\ \hline
 			$\rightarrow^* 00$ & $\emptyset  $ & $\{10\}$ & $\{10\}$ & $\{11\}$ \\  
 			$                 01$ & $\emptyset  $ & $\{11\}$ & $\{11\}$ & $\{10\}$ \\
 			$                 10$ & $\emptyset  $ & $\{00\}$ & $\{00\}$ & $\{01\}$\\
 			$                 11$ & $\emptyset  $ & $\{01\}$ & $\{01\}$ & $\{00\}$
		\end{tabular}
		\end{center}
	\end{minipage}
	\begin{minipage}[c]{0.27\textwidth}
		\begin{center}
			\begin{tikzpicture}[->,>=stealth',node distance=1.8cm,initial text=$ $,]
				\footnotesize
				\node[state, initial, accepting] (00) {$00$};
				\node[state, below of=00] (10) {$10$};
				\node[state, right of=00] (11) {$11$};
				\node[state, right of=10] (01) {$01$};
				
				\draw   (00) 	edge[bend right=10, left	] node{$a,b$} (10)
						(10) 	edge[bend right=10, right	] node{$a,b$} (00)
						
						(11) 	edge[bend right=10, left	] node{$a,b$} (01)
						(01) 	edge[bend right=10, right	] node{$a,b$} (11)
						
						(00) 	edge[bend right=10, below	] node{$c$} (11)
						(11) 	edge[bend right=10, above	] node{$c$} (00)
						
						(10) 	edge[bend right=10, below	] node{$c$} (01)
				        (01) 	edge[bend right=10, above	] node{$c$} (10)
						;
			\end{tikzpicture}
		\end{center}
	\end{minipage}
\end{center}
\subsection{Exercise 4}
\begin{center}
	\begin{minipage}[c]{0.4\textwidth}
		\begin{alignat*}{2}
			\varepsilon close(p)&=\{p,q,r\}\\
			\varepsilon close(q)&=\{q\}\\
			\varepsilon close(r)&=\{q,r\}\\
			\varepsilon close(s)&=\{s\}
		\end{alignat*}
	\end{minipage}%
	\begin{minipage}[c]{0.6\textwidth}
		\begin{center}
			\begin{tikzpicture}[->,>=stealth',node distance=2.5cm,initial text=$ $,]
				\footnotesize
				\node[state, initial, accepting] (pqr) {$\{p,q,r\}$};
				\node[state, accepting, below right of=pqr] (qrs) {$\{q,r,s\}$};
				\node[state, accepting, above right of=pqr] (pqrs) {$\{p,q,r,s\}$};
				
				\draw   (pqr) 	edge[bend right=10, below	] node{$a,d$} (pqrs)
						(pqr) 	edge[bend right=10, left	] node{$b$} (qrs)
						(pqr) 	edge[loop above				] node{$c$} (pqr)

						(qrs) 	edge[loop right				] node{$a$} (qrs)
						(qrs) 	edge[bend right=10, right	] node{$b,c$} (pqr)
						(qrs) 	edge[right					] node{$d$} (pqrs)

						(pqrs) 	edge[loop right				] node{$a,b,d$} (qrs)
						(pqrs) 	edge[bend right=10, left	] node{$c$} (pqr)
						;
			\end{tikzpicture}
		\end{center}
	\end{minipage}%
\end{center}
\begin{center}
	\begin{minipage}[c]{0.4\textwidth}
		\begin{alignat*}{2}
			DFA       &= (\Sigma, Q_D, \delta_D, \{p,q,r\}, F_D)\\
			\Sigma    &= \{a,b,c,d\}\\
			Q_D       &= \{\{p,q,r\},\{q,r,s\},\{p,q,r,s\}\}\\
			F_D       &= \{\{p,q,r\},\{q,r,s\},\{p,q,r,s\}\}\\
			\delta_D &\colon Q_D \times \Sigma \rightarrow Q_D
		\end{alignat*}
	\end{minipage}%
	\begin{minipage}[c]{0.6\textwidth}
		\begin{center}
		\begin{tabular}{ r | c c c c }
			$\delta_D                    $ & $a          $ & $b          $ & $c          $ & $d          $ \\ \hline
			$\rightarrow{^*} \{p,q,r  \}$ & $\{p,q,r,s\}$ & $\{  q,r,s\}$ & $\{p,q,r  \}$ & $\{p,q,r,s\}$ \\
			$           {^*} \{  q,r,s\}$ & $\{  q,r,s\}$ & $\{p,q,r  \}$ & $\{p,q,r  \}$ & $\{p,q,r,s\}$ \\
			$           {^*} \{p,q,r,s\}$ & $\{p,q,r,s\}$ & $\{p,q,r,s\}$ & $\{p,q,r  \}$ & $\{p,q,r,s\}$ \\
		\end{tabular}
		\end{center}
	\end{minipage}%
\end{center}
}
\end{document}
